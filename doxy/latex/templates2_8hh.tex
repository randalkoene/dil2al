\section{templates2.hh File Reference}
\label{templates2_8hh}\index{templates2.hh@{templates2.hh}}
\subsection*{Defines}
\begin{CompactItemize}
\item 
\#define {\bf Protected\_\-Linked\_\-List}({\bf PLLType})
\item 
\#define {\bf Protected\_\-Linked\_\-List\_\-Class}({\bf PLLType})\ public PLL\_\-PLLType
\item 
\#define {\bf Protected\_\-Linked\_\-List\_\-functions}({\bf PLLType})
\item 
\#define {\bf Protected\_\-Linked\_\-List\_\-Indexed\_\-Access}({\bf PLLType}, Here, There)
\item 
\#define {\bf PLL\_\-LOOP\_\-FORWARD}({\bf PLLType}, Init\_\-El, Extra\_\-Test)\ for ({\bf PLLType} $\ast$ e = Init\_\-El; ((e) \&\& (Extra\_\-Test)); e = e $\rightarrow$ Next())
\end{CompactItemize}
\subsection*{Functions}
\begin{CompactItemize}
\item 
{\bf for} (;(h);h=h $\rightarrow$ Next()) h $\rightarrow$ pllroot=this
\item 
void {\bf update} ()
\item 
bool PLLRoot\_\- {\bf if} (!{\bf succ}) return {\bf false}
\item 
{\bf if} (!{\bf succ} $\rightarrow$ head()) return {\bf false}
\item 
{\bf if} ({\bf plltail})
\end{CompactItemize}


\subsection{Define Documentation}
\index{templates2.hh@{templates2.hh}!PLL_LOOP_FORWARD@{PLL\_\-LOOP\_\-FORWARD}}
\index{PLL_LOOP_FORWARD@{PLL\_\-LOOP\_\-FORWARD}!templates2.hh@{templates2.hh}}
\subsubsection{\setlength{\rightskip}{0pt plus 5cm}\#define PLL\_\-LOOP\_\-FORWARD({\bf PLLType}, Init\_\-El, Extra\_\-Test)\ for ({\bf PLLType} $\ast$ e = Init\_\-El; ((e) \&\& (Extra\_\-Test)); e = e $\rightarrow$ Next())}\label{templates2_8hh_a4}


\index{templates2.hh@{templates2.hh}!Protected_Linked_List@{Protected\_\-Linked\_\-List}}
\index{Protected_Linked_List@{Protected\_\-Linked\_\-List}!templates2.hh@{templates2.hh}}
\subsubsection{\setlength{\rightskip}{0pt plus 5cm}\#define Protected\_\-Linked\_\-List({\bf PLLType})}\label{templates2_8hh_a0}




Definition at line 48 of file templates2.hh.\index{templates2.hh@{templates2.hh}!Protected_Linked_List_Class@{Protected\_\-Linked\_\-List\_\-Class}}
\index{Protected_Linked_List_Class@{Protected\_\-Linked\_\-List\_\-Class}!templates2.hh@{templates2.hh}}
\subsubsection{\setlength{\rightskip}{0pt plus 5cm}\#define Protected\_\-Linked\_\-List\_\-Class({\bf PLLType})\ public PLL\_\-PLLType}\label{templates2_8hh_a1}




Definition at line 93 of file templates2.hh.\index{templates2.hh@{templates2.hh}!Protected_Linked_List_functions@{Protected\_\-Linked\_\-List\_\-functions}}
\index{Protected_Linked_List_functions@{Protected\_\-Linked\_\-List\_\-functions}!templates2.hh@{templates2.hh}}
\subsubsection{\setlength{\rightskip}{0pt plus 5cm}\#define Protected\_\-Linked\_\-List\_\-functions({\bf PLLType})}\label{templates2_8hh_a2}




Definition at line 99 of file templates2.hh.\index{templates2.hh@{templates2.hh}!Protected_Linked_List_Indexed_Access@{Protected\_\-Linked\_\-List\_\-Indexed\_\-Access}}
\index{Protected_Linked_List_Indexed_Access@{Protected\_\-Linked\_\-List\_\-Indexed\_\-Access}!templates2.hh@{templates2.hh}}
\subsubsection{\setlength{\rightskip}{0pt plus 5cm}\#define Protected\_\-Linked\_\-List\_\-Indexed\_\-Access({\bf PLLType}, Here, There)}\label{templates2_8hh_a3}


{\bf Value:}

\footnotesize\begin{verbatim}if (!n) return Here; \
        if (PLLType * e = el(n)) return e->There; else return NULL\end{verbatim}\normalsize 


\subsection{Function Documentation}
\index{templates2.hh@{templates2.hh}!for@{for}}
\index{for@{for}!templates2.hh@{templates2.hh}}
\subsubsection{\setlength{\rightskip}{0pt plus 5cm}for (;(h); {\em h} = h-, Next())}\label{templates2_8hh_a5}


\index{templates2.hh@{templates2.hh}!if@{if}}
\index{if@{if}!templates2.hh@{templates2.hh}}
\subsubsection{\setlength{\rightskip}{0pt plus 5cm}if ({\bf plltail})}\label{templates2_8hh_a9}




Definition at line 183 of file templates2.hh.

References PLLType$<$ PLLType $>$::head(), PLLType$<$ PLLType $>$::pllprev, plltail, and succ.



\footnotesize\begin{verbatim}183                              {
184                         plltail->pllnext = succ->head();
185                         succ->head()->pllprev = plltail;
186                 } else pllhead = succ->head();
\end{verbatim}\normalsize 
\index{templates2.hh@{templates2.hh}!if@{if}}
\index{if@{if}!templates2.hh@{templates2.hh}}
\subsubsection{\setlength{\rightskip}{0pt plus 5cm}if (!{\bf succ}-, head())}\label{templates2_8hh_a8}


\index{templates2.hh@{templates2.hh}!if@{if}}
\index{if@{if}!templates2.hh@{templates2.hh}}
\subsubsection{\setlength{\rightskip}{0pt plus 5cm}bool PLLRoot\_\- if (! {\em succ})\hspace{0.3cm}{\tt  [inline]}}\label{templates2_8hh_a7}


\index{templates2.hh@{templates2.hh}!update@{update}}
\index{update@{update}!templates2.hh@{templates2.hh}}
\subsubsection{\setlength{\rightskip}{0pt plus 5cm}void update ()\hspace{0.3cm}{\tt  [inline]}}\label{templates2_8hh_a6}




Definition at line 164 of file templates2.hh.

References pllhead, and plltail.



\footnotesize\begin{verbatim}164                              {
165         // clears head and tail if elements were linked to another root
166         // or fixes head and tail if either one pointed to an element with
167         // a different root
168                 if (pllhead) if (pllhead->Root()!=this) pllhead = NULL;
169                 if (plltail) {
170                         if (plltail->Root()!=this) {
171                                 if (!pllhead) plltail = NULL;
172                                 else plltail = pllhead->seek_tail();
173                         } else {
174                                 if (!pllhead) pllhead = plltail->seek_head();
175                         }
176                 }
177         }
\end{verbatim}\normalsize 
